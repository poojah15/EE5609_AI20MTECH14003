\documentclass[journal,12pt,twocolumn]{IEEEtran}

\usepackage{setspace}
\usepackage{gensymb}

\singlespacing
\usepackage{pstricks}
\usepackage[cmex10]{amsmath}
%\usepackage{amsthm}
%\interdisplaylinepenalty=2500
%\savesymbol{iint}
%\usepackage{txfonts}
%\restoresymbol{TXF}{iint}
%\usepackage{wasysym}
\usepackage{amsthm}
%\usepackage{iithtlc}
\usepackage{mathrsfs}
\usepackage{txfonts}
\usepackage{stfloats}
\usepackage{bm}
\usepackage{cite}
\usepackage{cases}
\usepackage{subfig}
%\usepackage{xtab}
\usepackage{longtable}
\usepackage{multirow}
%\usepackage{algorithm}
%\usepackage{algpseudocode}
\usepackage{enumitem}
\usepackage{mathtools}
\usepackage{caption}
\usepackage{steinmetz}
\usepackage{tikz}
%\usepackage{circuitikz}
\usepackage{verbatim}
\usepackage{tfrupee}
\usepackage[breaklinks=true]{hyperref}
%\usepackage{stmaryrd}
\usepackage{tkz-euclide} % loads  TikZ and tkz-base
%\usetkzobj{all}
\usetikzlibrary{calc,math}
\usepackage{listings}
\usepackage{color}                                            %%
\usepackage{array}                                            %%
\usepackage{longtable}                                        %%
\usepackage{calc}                                             %%
\usepackage{multirow}                                         %%
\usepackage{hhline}                                           %%
\usepackage{ifthen}                                           %%
%optionally (for landscape tables embedded in another document): %%
\usepackage{lscape}     
%\usepackage{multicol}
\usepackage{chngcntr}
%\usepackage{enumerate}
%\usepackage{wasysym}
%\newcounter{MYtempeqncnt}
\DeclareMathOperator*{\Res}{Res}
%\renewcommand{\baselinestretch}{2}
\renewcommand\thesection{\arabic{section}}
\renewcommand\thesubsection{\thesection.\arabic{subsection}}
\renewcommand\thesubsubsection{\thesubsection.\arabic{subsubsection}}

\renewcommand\thesectiondis{\arabic{section}}
\renewcommand\thesubsectiondis{\thesectiondis.\arabic{subsection}}
\renewcommand\thesubsubsectiondis{\thesubsectiondis.\arabic{subsubsection}}

% correct bad hyphenation here
\hyphenation{op-tical net-works semi-conduc-tor}
\def\inputGnumericTable{}                                 %%

\lstset{
	%language=C,
	frame=single, 
	breaklines=true,
	columns=fullflexible
}
\usepackage{amssymb}
\usepackage{stackengine}
\usepackage{scalerel}
\usepackage{graphicx}
\newlength\lthk
\setlength\lthk{.1ex}
\def\bline{\rule{2ex}{\lthk}}
\def\slash{\rotatebox{60}{\bline}}
\def\parallelogram{\stackMath\scalerel*{%
		\def\stackalignment{l}{\stackunder[-.5\lthk]{%
				\def\stackalignment{r}\stackon[-.5\lthk]{\slash\rule{.866ex}{0ex}\slash}{\bline}}%
			{\bline}}}{\square}%
}

\begin{document}
	%
	
	
	\newtheorem{theorem}{Theorem}[section]
	\newtheorem{problem}{Problem}
	\newtheorem{proposition}{Proposition}[section]
	\newtheorem{lemma}{Lemma}[section]
	\newtheorem{corollary}[theorem]{Corollary}
	\newtheorem{example}{Example}[section]
	\newtheorem{definition}[problem]{Definition}
	
	\newcommand{\BEQA}{\begin{eqnarray}}
		\newcommand{\EEQA}{\end{eqnarray}}
	\newcommand{\define}{\stackrel{\triangle}{=}}
	\bibliographystyle{IEEEtran}
	%\bibliographystyle{ieeetr}
	\providecommand{\mbf}{\mathbf}
	\providecommand{\pr}[1]{\ensuremath{\Pr\left(#1\right)}}
	\providecommand{\qfunc}[1]{\ensuremath{Q\left(#1\right)}}
	\providecommand{\sbrak}[1]{\ensuremath{{}\left[#1\right]}}
	\providecommand{\lsbrak}[1]{\ensuremath{{}\left[#1\right.}}
	\providecommand{\rsbrak}[1]{\ensuremath{{}\left.#1\right]}}
	\providecommand{\brak}[1]{\ensuremath{\left(#1\right)}}
	\providecommand{\lbrak}[1]{\ensuremath{\left(#1\right.}}
	\providecommand{\rbrak}[1]{\ensuremath{\left.#1\right)}}
	\providecommand{\cbrak}[1]{\ensuremath{\left\{#1\right\}}}
	\providecommand{\lcbrak}[1]{\ensuremath{\left\{#1\right.}}
	\providecommand{\rcbrak}[1]{\ensuremath{\left.#1\right\}}}
	\theoremstyle{remark}
	\newtheorem{rem}{Remark}
	\newcommand{\sgn}{\mathop{\mathrm{sgn}}}
	\providecommand{\abs}[1]{\left\vert#1\right\vert}
	\providecommand{\res}[1]{\Res\displaylimits_{#1}} 
	\providecommand{\norm}[1]{\left\lVert#1\right\rVert}
	%\providecommand{\norm}[1]{\lVert#1\rVert}
	\providecommand{\mtx}[1]{\mathbf{#1}}
	\providecommand{\mean}[1]{E\left[ #1 \right]}
	\providecommand{\fourier}{\overset{\mathcal{F}}{ \rightleftharpoons}}
	%\providecommand{\hilbert}{\overset{\mathcal{H}}{ \rightleftharpoons}}
	\providecommand{\system}{\overset{\mathcal{H}}{ \longleftrightarrow}}
	%\newcommand{\solution}[2]{\textbf{Solution:}{#1}}
	\newcommand{\solution}{\noindent \textbf{Solution: }}
	\newcommand{\cosec}{\,\text{cosec}\,}
	\providecommand{\dec}[2]{\ensuremath{\overset{#1}{\underset{#2}{\gtrless}}}}
	\newcommand{\myvec}[1]{\ensuremath{\begin{pmatrix}#1\end{pmatrix}}}
	\newcommand{\mydet}[1]{\ensuremath{\begin{vmatrix}#1\end{vmatrix}}}
	%\numberwithin{equation}{section}
	\numberwithin{equation}{subsection}
	%\numberwithin{problem}{section}
	%\numberwithin{definition}{section}
	\makeatletter
	\@addtoreset{figure}{problem}
	\makeatother
	\let\StandardTheFigure\thefigure
	\let\vec\mathbf
	%\renewcommand{\thefigure}{\theproblem.\arabic{figure}}
	\renewcommand{\thefigure}{\theproblem}
	%\setlist[enumerate,1]{before=\renewcommand\theequation{\theenumi.\arabic{equation}}
	%\counterwithin{equation}{enumi}
	%\renewcommand{\theequation}{\arabic{subsection}.\arabic{equation}}
	\def\putbox#1#2#3{\makebox[0in][l]{\makebox[#1][l]{}\raisebox{\baselineskip}[0in][0in]{\raisebox{#2}[0in][0in]{#3}}}}
	\def\rightbox#1{\makebox[0in][r]{#1}}
	\def\centbox#1{\makebox[0in]{#1}}
	\def\topbox#1{\raisebox{-\baselineskip}[0in][0in]{#1}}
	\def\midbox#1{\raisebox{-0.5\baselineskip}[0in][0in]{#1}}
	\vspace{3cm}
	\title{Assignment-13}
	\author{Pooja H \\ AI20MTECH14003}
	\maketitle
	\newpage
	\bigskip
	\renewcommand{\thefigure}{\theenumi}
	\renewcommand{\thetable}{\theenumi}
\begin{abstract}
	In this document, we find whether the given function T from $\vec{R}^2$ into $\vec{R}^2$ is a linear transformation or not.
\end{abstract}
%Download all python codes from 
%\begin{lstlisting}
%	https://github.com/poojah15/EE5609_AI20MTECH14003/tree/master/Assignment_13
%\end{lstlisting}
Download all latex-tikz codes from 
\begin{lstlisting}
	https://github.com/poojah15/EE5609_AI20MTECH14003/tree/master/Assignment_13
\end{lstlisting}
\section{Problem Statement}
Verify whether $T(x_1, x_2)  = (x_1 - x_2, 0)$ is a linear transformation or not.

\section{Solution}
Let $\vec{V}$ and $\vec{W}$ be the vector spaces.  The function $T : \vec{V} \rightarrow \vec{W}$ is called a linear transformation of $\vec{V}$ into $\vec{W}$ if the following properties are true for all $\vec{u}$ and $\vec{v}$ in $\vec{V}$ and for any scalar $k$.
\begin{align}
	T(\vec{u} + \vec{v}) &= T(\vec{u}) + T(\vec{v})\\
	T(k\vec{u}) &= kT(\vec{u})
\end{align}
Given, 
\begin{align}
	T \myvec{x_1 \\x_2}  = \myvec{1 & -1\\0 & 0} \myvec{x_1 \\ x_2}
\end{align}
Suppose we have $\vec{x} = \myvec{x_1\\x_2}$ and $\vec{y} = \myvec{y_1\\ y_2}$, then we have to show that the result of $T(\vec{x}+\vec{y})$ must be same as $T(\vec{x}) + T(\vec{y})$. Consider,
\begin{align}
	T(\vec{x} + \vec{y}) &= T\myvec{x_1 + y_1\\ x_2 + y_2}\\
	&= \myvec{1  & -1\\0 & 0}\myvec{x_1 + y_1 \\ x_2 + y_2}\\
	&= \myvec{x_1 + y_1 - x_2 - y_2\\0}\\
    &= \myvec{x_1 - x_2\\0} + \myvec{y_1 - y_2\\0}\\
    &=\myvec{1 & -1\\0 &0}\myvec{x_1 \\x_2} + \myvec{1 & -1\\0 &0}\myvec{y_1 \\y_2} \\
  T(\vec{x} + \vec{y})  &= T(\vec{x}) + T(\vec{y})
\end{align}
Also, we need to show that for some constant $k$, $T(k\vec{x}) = kT(\vec{x})$. Consider,
\begin{align}
	T(k\vec{x}) &= T\myvec{kx_1\\kx_2}\\
	&= \myvec{k & -k \\0 & 0} \myvec{x_1 \\x_2}\\
	&= k\myvec{1 & -1 \\0 & 0} \myvec{x_1 \\x_2}\\
	T(k\vec{x}) &= kT(\vec{x})
\end{align}
Therefore, the given function T is a linear transformation.
\end{document}
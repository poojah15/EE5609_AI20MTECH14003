\documentclass[a4paper,12pt]{article}
\usepackage{amsmath, mathtools}
\usepackage{extarrows}
\newcommand{\myvec}[1]{\ensuremath{\begin{pmatrix}#1\end{pmatrix}}}

\begin{document}
\begin{flushleft}	
\textbf{Problem Statement:} \\
\textit{To verify whether the lines passing through the given set of points are parallel or not} \vspace{5mm}

\textbf{Theory:} \\
Let the lines be parallel and the first two points pass through $n^T\textbf{x} = c1$\\
i.e.
\begin{align}
n^Tx_1=c_1 => x_1^Tn = c_1 \\
n^Tx_2=c_2 => x_2^Tn = c_2
\end{align}
and the second two points pass through $n^T\textbf{x} = c2$
Then
\begin{align}
n^Tx_3=c_3 => x_3^Tn = c_3 \\
n^Tx_4=c_4 => x_4^Tn = c_4
\end{align}
Putting all the equations together, we obtain
\begin{align}
\myvec{x_1^T\\x_2^T\\x_3^T\\x_4^T}\vec{n} = \myvec{c_1\\c_2\\c_3\\c_4}
\end{align}
Now if this equation has a solution, then $\vec{n}$ exists and the lines will be parallel.\\
\vspace{3mm}
\textbf{Example:}\\
Given the points, $\textbf{A} = \begin{pmatrix} 4\\	7\\	8\end{pmatrix}, 
\textbf{B} = \begin{pmatrix} 2\\ 3\\ 4 \end{pmatrix}, and \quad
\textbf{C} = \begin{pmatrix} -1\\ -2\\ 1 \end{pmatrix}, 
\textbf{D} = \begin{pmatrix} 1\\ 2\\ 5 \end{pmatrix} $\\

Applying the row reduction procedure on the coefficient matrix:
\begin{align*}
\begin{pmatrix} 4 & 7 & 8\\ 2 & 3 & 4\\ -1 & -2 & 1\\ 1 & 2 & 5 \end{pmatrix}
\underset{\overset{r_3+r_4}{\longrightarrow}}{\overset{r_1 - 2r_2}{\longrightarrow}}
\begin{pmatrix} 4 & 7 & 8\\ 0 & 1 & 0\\ -1 & -2 & 1\\ 0 & 0 & 6 \end{pmatrix}
\underset{\overset{r_3-6r_4}{\longrightarrow}}{\overset{r_1 - 7r_2}{\longrightarrow}}
\begin{pmatrix} 4 & 0 & 8\\ 0 & 1 & 0\\ -1 & -2 & 0\\ 0 & 0 & 6 \end{pmatrix}
\underset{\overset{r_1-8r_4}{\longrightarrow}}{\overset{r_4/6}{\longrightarrow}}
\begin{pmatrix} 4 & 0 & 0\\ 0 & 1 & 0\\ -1 & -2 & 1\\ 0 & 0 & 1 \end{pmatrix} \\
\underset{\overset{r_3+r_4}{\longrightarrow}}{\overset{-r_3 - 2r_2}{\longrightarrow}}
\begin{pmatrix} 4 & 0 & 0\\ 0 & 1 & 0\\ 1 & 0 & 0\\ 0 & 0 & 1 \end{pmatrix}
{\overset{r_1 - 4r_3}{\longrightarrow}}
\begin{pmatrix} 0 & 0 & 0\\ 0 & 1 & 0\\ 1 & 0 & 0\\ 0 & 0 & 1 \end{pmatrix}
\end{align*}
Here, the number of non-zero rows are three and hence the rank of the matrix is 3 which implies that the solution exists. Therefore the lines passing through A, B and C, D are parallel.

\end{flushleft}
\end{document}
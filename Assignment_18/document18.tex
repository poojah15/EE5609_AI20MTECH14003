\documentclass[journal,12pt,twocolumn]{IEEEtran}

\usepackage{setspace}
\usepackage{gensymb}

\singlespacing
\usepackage{pstricks}
\usepackage[cmex10]{amsmath}
%\usepackage{amsthm}
%\interdisplaylinepenalty=2500
%\savesymbol{iint}
%\usepackage{txfonts}
%\restoresymbol{TXF}{iint}
%\usepackage{wasysym}
\usepackage{amsthm}
%\usepackage{iithtlc}
\usepackage{mathrsfs}
\usepackage{txfonts}
\usepackage{stfloats}
\usepackage{bm}
\usepackage{cite}
\usepackage{cases}
\usepackage{subfig}
%\usepackage{xtab}
\usepackage{longtable}
\usepackage{multirow}
%\usepackage{algorithm}
%\usepackage{algpseudocode}
\usepackage{enumitem}
\usepackage{mathtools}
\usepackage{caption}
\usepackage{steinmetz}
\usepackage{tikz}
%\usepackage{circuitikz}
\usepackage{verbatim}
\usepackage{tfrupee}
\usepackage[breaklinks=true]{hyperref}
%\usepackage{stmaryrd}
\usepackage{tkz-euclide} % loads  TikZ and tkz-base
%\usetkzobj{all}
\usetikzlibrary{calc,math}
\usepackage{listings}
\usepackage{color}                                            %%
\usepackage{array}                                            %%
\usepackage{longtable}                                        %%
\usepackage{calc}                                             %%
\usepackage{multirow}                                         %%
\usepackage{hhline}                                           %%
\usepackage{ifthen}                                           %%
%optionally (for landscape tables embedded in another document): %%
\usepackage{lscape}     
%\usepackage{multicol}
\usepackage{chngcntr}
\usepackage{makecell}
\usepackage{booktabs, threeparttable}
\usepackage{lipsum}
%\usepackage{enumerate}
%\usepackage{wasysym}
%\newcounter{MYtempeqncnt}
\DeclareMathOperator*{\Res}{Res}
%\renewcommand{\baselinestretch}{2}
\renewcommand\thesection{\arabic{section}}
\renewcommand\thesubsection{\thesection.\arabic{subsection}}
\renewcommand\thesubsubsection{\thesubsection.\arabic{subsubsection}}

\renewcommand\thesectiondis{\arabic{section}}
\renewcommand\thesubsectiondis{\thesectiondis.\arabic{subsection}}
\renewcommand\thesubsubsectiondis{\thesubsectiondis.\arabic{subsubsection}}

% correct bad hyphenation here
\hyphenation{op-tical net-works semi-conduc-tor}
\def\inputGnumericTable{}                                 %%
\newcommand\myemptypage{
	\null
	\thispagestyle{empty}
	\addtocounter{page}{-1}
	\newpage
}
\lstset{
	%language=C,
	frame=single, 
	breaklines=true,
	columns=fullflexible
}
\usepackage{amssymb}
\usepackage{stackengine}
\usepackage{scalerel}
\usepackage{graphicx}
\usepackage{tabularx, ragged2e}
\newcolumntype{C}{>{ \Centering \arraybackslash}X} %
\frenchspacing
\newlength\lthk
\setlength\lthk{.1ex}
\def\bline{\rule{2ex}{\lthk}}
\def\slash{\rotatebox{60}{\bline}}
\def\parallelogram{\stackMath\scalerel*{%
		\def\stackalignment{l}{\stackunder[-.5\lthk]{%
				\def\stackalignment{r}\stackon[-.5\lthk]{\slash\rule{.866ex}{0ex}\slash}{\bline}}%
			{\bline}}}{\square}%
}

\begin{document}
	%
	
	
	\newtheorem{theorem}{Theorem}[section]
	\newtheorem{problem}{Problem}
	\newtheorem{proposition}{Proposition}[section]
	\newtheorem{lemma}{Lemma}[section]
	\newtheorem{corollary}[theorem]{Corollary}
	\newtheorem{example}{Example}[section]
	\newtheorem{definition}[problem]{Definition}
	
	\newcommand{\BEQA}{\begin{eqnarray}}
		\newcommand{\EEQA}{\end{eqnarray}}
	\newcommand{\define}{\stackrel{\triangle}{=}}
	\bibliographystyle{IEEEtran}
	%\bibliographystyle{ieeetr}
	\providecommand{\mbf}{\mathbf}
	\providecommand{\pr}[1]{\ensuremath{\Pr\left(#1\right)}}
	\providecommand{\qfunc}[1]{\ensuremath{Q\left(#1\right)}}
	\providecommand{\sbrak}[1]{\ensuremath{{}\left[#1\right]}}
	\providecommand{\lsbrak}[1]{\ensuremath{{}\left[#1\right.}}
	\providecommand{\rsbrak}[1]{\ensuremath{{}\left.#1\right]}}
	\providecommand{\brak}[1]{\ensuremath{\left(#1\right)}}
	\providecommand{\lbrak}[1]{\ensuremath{\left(#1\right.}}
	\providecommand{\rbrak}[1]{\ensuremath{\left.#1\right)}}
	\providecommand{\cbrak}[1]{\ensuremath{\left\{#1\right\}}}
	\providecommand{\lcbrak}[1]{\ensuremath{\left\{#1\right.}}
	\providecommand{\rcbrak}[1]{\ensuremath{\left.#1\right\}}}
	\theoremstyle{remark}
	\newtheorem{rem}{Remark}
	\newcommand{\sgn}{\mathop{\mathrm{sgn}}}
	\providecommand{\abs}[1]{\left\vert#1\right\vert}
	\providecommand{\res}[1]{\Res\displaylimits_{#1}} 
	\providecommand{\norm}[1]{\left\lVert#1\right\rVert}
	%\providecommand{\norm}[1]{\lVert#1\rVert}
	\providecommand{\mtx}[1]{\mathbf{#1}}
	\providecommand{\mean}[1]{E\left[ #1 \right]}
	\providecommand{\fourier}{\overset{\mathcal{F}}{ \rightleftharpoons}}
	%\providecommand{\hilbert}{\overset{\mathcal{H}}{ \rightleftharpoons}}
	\providecommand{\system}{\overset{\mathcal{H}}{ \longleftrightarrow}}
	%\newcommand{\solution}[2]{\textbf{Solution:}{#1}}
	\newcommand{\solution}{\noindent \textbf{Solution: }}
	\newcommand{\cosec}{\,\text{cosec}\,}
	\providecommand{\dec}[2]{\ensuremath{\overset{#1}{\underset{#2}{\gtrless}}}}
	\newcommand{\myvec}[1]{\ensuremath{\begin{pmatrix}#1\end{pmatrix}}}
	\newcommand{\mydet}[1]{\ensuremath{\begin{vmatrix}#1\end{vmatrix}}}
	%\numberwithin{equation}{section}
	\numberwithin{equation}{subsection}
	%\numberwithin{problem}{section}
	%\numberwithin{definition}{section}
	\makeatletter
	\@addtoreset{figure}{problem}
	\makeatother
	\let\StandardTheFigure\thefigure
	\let\vec\mathbf
	%\renewcommand{\thefigure}{\theproblem.\arabic{figure}}
	\renewcommand{\thefigure}{\theproblem}
	%\setlist[enumerate,1]{before=\renewcommand\theequation{\theenumi.\arabic{equation}}
	%\counterwithin{equation}{enumi}
	%\renewcommand{\theequation}{\arabic{subsection}.\arabic{equation}}
	\def\putbox#1#2#3{\makebox[0in][l]{\makebox[#1][l]{}\raisebox{\baselineskip}[0in][0in]{\raisebox{#2}[0in][0in]{#3}}}}
	\def\rightbox#1{\makebox[0in][r]{#1}}
	\def\centbox#1{\makebox[0in]{#1}}
	\def\topbox#1{\raisebox{-\baselineskip}[0in][0in]{#1}}
	\def\midbox#1{\raisebox{-0.5\baselineskip}[0in][0in]{#1}}
	\vspace{3cm}
	\title{Assignment-18}
	\author{Pooja H \\ AI20MTECH14003}
	\maketitle
	\newpage
	\bigskip
	\renewcommand{\thefigure}{\theenumi}
	\renewcommand{\thetable}{\theenumi}
	\begin{abstract}
		In this document, we explore the properties of eigenvalues of similar matrices.
	\end{abstract}
	%Download all python codes from 
	%\begin{lstlisting}
	%	https://github.com/poojah15/EE5609_AI20MTECH14003/tree/master/Assignment_13
	%\end{lstlisting}
	Download all latex-tikz codes from 
\begin{lstlisting}
https://github.com/poojah15/EE5609_AI20MTECH14003/tree/master/Assignment_18
\end{lstlisting}
	\section{Problem Statement}

Let $\vec{A}$ and $\vec{B}$ be $n \times n$ matrices over $\vec{C}$. Then,
\begin{enumerate}
	\item $\vec{AB}$ and $\vec{BA}$ always have the same set of eigenvalues.
	\item If $\vec{AB}$ and $\vec{BA}$ have the same set of eigenvalues then $\vec{AB=BA}$
	\item If $\vec{A}^{-1}$ exists, then $\vec{AB}$ and $\vec{BA}$ are similar
	\item The rank of $\vec{AB}$  is always the same as the rank of $\vec{BA}$.
\end{enumerate}

\section{Answers for different cases}
\pagebreak
\myemptypage
\begin{table}[h]
	\begin{tabular}{|m{3cm}|m{14cm}|}
		\hline
		&\\
		$\vec{AB}$ and $\vec{BA}$ always have the same set of eigenvalues.
		& \textbf{True}. \\
		& Let  $\lambda$  be an eigenvalue of  $\vec{AB}$, and $\vec{x}$  be a corresponding eigenvector.\\
		&Then \\
		& \qquad\qquad\qquad$\vec{AB} \vec{x}$=$\lambda \vec{x}$ \\
		& Left-multiplying by $\vec{B}$:\\
		&\qquad\qquad\qquad$\vec{B(AB)}\vec{x}$ = $\vec{B}(\lambda \vec{x})$\\
		& \qquad\qquad\qquad$\vec{(BA)}\vec{Bx}=\lambda(\vec{B}\vec{x})$  (by associativity of multiplication)\\
	    & $\implies  \lambda $ is an eigenvalue of  $\vec{BA}$  with  $\vec{B}\vec{x}$  as the corresponding eigenvector, assuming  $\vec{B}\vec{x}$  is not a null vector.\\
		&If  $\vec{B}\vec{x}$   is null, then  $\vec{B}$   is singular, so that both  $\vec{AB}$ and  $\vec{BA}$ are singular, and  $\lambda=0$. Since both the products are singular,  0  is an eigenvalue of both.\\
		&\\
		& Example:\\
		& Let \\
		&\qquad\qquad\qquad$\vec{A} = \myvec{2 & 0 \\ 0 & 3}, \vec{B} = \myvec{1 & 0 \\ 0 & 4}$\\
		&Then\\
		& \qquad\qquad\qquad$\vec{AB} = \myvec{2 & 0 \\ 0 & 12}, \vec{BA} = \myvec{2 & 0 \\ 0 & 12}$\\
		& Since $\vec{AB}$ and $\vec{BA}$ results with the same characteristic equation,\\
		& \qquad\qquad\qquad $\lambda^2 - 14\lambda + 24$\\
		&they will have same set of eigenvalues that is $\lambda_1 = 2, \lambda_2 = 12$\\
		& \\
		\hline
		&\\
		If $\vec{AB}$ and $\vec{BA}$ have the same set of eigenvalues then $\vec{AB=BA}$
		& \textbf{False}. \\
		&Counter example:\\
		& Let\\
		& \qquad\qquad\qquad$\vec{A}$ = $\myvec{0 & 1 \\ 0 & 0}$ , 
	     $\vec{B}$ = $\myvec{1 & 0 \\ 0 & 0}$\\
	    &then\\
	    & \qquad\qquad\qquad$\vec{AB}$ = $\myvec{0 & 0 \\ 0 & 0}$ ,  $\vec{BA}$ = $\myvec{0 & 1 \\ 0 & 0}$\\
        & $\implies$ Similar eigenvalues, but $\vec{AB} \ne \vec{BA}$
		\\
		&\\
		\hline
			\end{tabular}
	\end{table}
\pagebreak
\myemptypage
\begin{table}[h]
	\begin{tabular}{|m{3cm}|m{14cm}|}
		\hline
		&\\
	    If $\vec{A}^{-1}$ exists, then $\vec{AB}$ and $\vec{BA}$ are similar
		& \textbf{True}. \\
		& Given that $\vec{A}^{-1}$ exists and hence,\\
		&\qquad\qquad\qquad $\vec{AB}$ = $\vec{A}^{-1}\vec{(AB)A}$ =  $\vec{(A^{-1}A)}\vec{BA}$ = $\vec{BA}$.\\
		& Hence, $\vec{AB} \simeq \vec{BA}$ \\
		&\\
		&Example:\\
		& Let\\
		& \qquad\qquad\qquad$\vec{A}$ = $\myvec{2 & 5 \\ 1 & 3}$ , 
		$\vec{B}$ = $\myvec{3 & 1 \\ 2 & 4}$\\
		&then\\
		& \qquad\qquad\qquad$\vec{AB}$\quad= $\myvec{16 & 22 \\ 9 & 13} = \vec{A}^{-1} (\vec{AB})\vec{A}$\\
		&  \qquad\qquad\qquad  \qquad\qquad\qquad\quad= $\myvec{3 & -5\\ -1 & 2}\myvec{16 & 22 \\ 9 & 13}\myvec{2 & 5 \\ 1 & 3}$\\
		&  \qquad\qquad\qquad \qquad\qquad\qquad\quad=$\myvec{7 & 18 \\ 8 & 22} $\\
		&  \qquad\qquad\qquad \qquad\qquad\qquad\quad= $\vec{BA}$\\
		&\\
		\hline
		&\\
	    The rank of $\vec{AB}$  is always the same as the rank of $\vec{BA}$.
		& \textbf{False}. \\
		&Counter example:\\
		& Let\\
		& \qquad\qquad\qquad$\vec{A}$ = $\myvec{0 & 1 \\ 0 & 0}$ , 
		$\vec{B}$ = $\myvec{1 & 0 \\ 0 & 0}$\\
		&then\\
		& \qquad\qquad\qquad$\vec{AB}$ = $\myvec{0 & 0 \\ 0 & 0}$ ,  $\vec{BA}$ = $\myvec{0 & 1 \\ 0 & 0}$\\
		& From the above $\vec{AB}$ and $\vec{BA}$, it is noted that the rank$\brak{\vec{AB}}$ = 0 and rank$\brak{\vec{BA}}$=1. \\
		& Hence the rank of $\vec{AB}$ need not always be same as rank of $\vec{BA}$.
		\\
		\hline
	\end{tabular}
\end{table}
\end{document}

\documentclass[a4paper,12pt]{article}
\usepackage{amsmath, mathtools}
\usepackage{extarrows}

\begin{document}
\begin{flushleft}	
\textbf{Problem Statement:} \\
\textit{To verify whether the lines passing through the given set of points are parallel or not} \vspace{5mm}

\textbf{Solution-1}
\textit{Using the vector representation} \\
Given the points, $A = \begin{pmatrix} 4\\	7\\	8\end{pmatrix}, 
                   B = \begin{pmatrix} 2\\ 3\\ 4 \end{pmatrix}, and \quad
                   C = \begin{pmatrix} -1\\ -2\\ 1 \end{pmatrix}, 
                   D = \begin{pmatrix} 1\\ 2\\ 5 \end{pmatrix} $


\textbullet{Compute the direction vector for the given set of points}
\begin{align}
B-A = \begin{pmatrix} -2\\ -4\\ -4 \end{pmatrix}   \\
D-C = \begin{pmatrix}  2\\  4\\  4 \end{pmatrix}  
\end{align}

\textbullet{Check whether one of the direction vector is the scalar multiple of the other direction vector}\\
Here, from (1) and (2), $B-A = k (D-C)$. In this example, k = -1.
\textbf{Hence, the lines are parallel.}\\
\vspace{5mm}
\textbf{Solution-2} \textit{Using the matrix representation and rank of a matrix}\\
Represent the direction vectors in the matrix form and perform row reduction: 
\begin{align*}
i.e., M = (B-A \quad D-C)^T \\
M = \begin{pmatrix}
	-2 & -4 & -4\\
	 2 &  4 &  4
    \end{pmatrix}
\xleftrightarrow{R_2 \leftarrow R_1+R_2}
\begin{pmatrix}
	-2 & -4 & -4\\
	 0 &  0 &  0
\end{pmatrix}
\end{align*}

Here, the rank of the matrix is 1. This implies that the lines are parallel.\\
\vspace{5mm}
\textbf{Solution-3} \textit{Using the cross product of the vectors}\\
\textbullet{Compute the cross product of the direction vectors}\\
The cross product of the direction vectors given in (1) and (2) is: 
\begin{align*}
\begin{pmatrix} -2\\ -4\\ -4 \end{pmatrix} X
\begin{pmatrix}  2\\  4\\  4 \end{pmatrix} =
\begin{pmatrix}
 -16 + 16 \\
 -8+8 \\
 -8+8\\
\end{pmatrix}
=
\begin{pmatrix}
0 \\
0 \\
0\\
\end{pmatrix}
\end{align*}

The zero vector infers that the lines are parallel.\\
\vspace{5mm}
\textbf{Solution-4} \textit{ Applying row reduction method on points represented in the form of matrix}\\

\begin{align*}
\begin{pmatrix} 4 & 7 & 8\\ 2 & 3 & 4\\ -1 & -2 & 1\\ 1 & 2 & 5 \end{pmatrix}
\underset{\overset{r_3+r_4}{\longrightarrow}}{\overset{r_1 - 2r_2}{\longrightarrow}}
\begin{pmatrix} 4 & 7 & 8\\ 0 & 1 & 0\\ -1 & -2 & 1\\ 0 & 0 & 6 \end{pmatrix}
\underset{\overset{r_3-6r_4}{\longrightarrow}}{\overset{r_1 - 7r_2}{\longrightarrow}}
\begin{pmatrix} 4 & 0 & 8\\ 0 & 1 & 0\\ -1 & -2 & 0\\ 0 & 0 & 6 \end{pmatrix}
\underset{\overset{r_1-8r_4}{\longrightarrow}}{\overset{r_4/6}{\longrightarrow}}
\begin{pmatrix} 4 & 0 & 0\\ 0 & 1 & 0\\ -1 & -2 & 1\\ 0 & 0 & 1 \end{pmatrix} \\
\underset{\overset{r_3+r_4}{\longrightarrow}}{\overset{-r_3 - 2r_2}{\longrightarrow}}
\begin{pmatrix} 4 & 0 & 0\\ 0 & 1 & 0\\ 1 & 0 & 0\\ 0 & 0 & 1 \end{pmatrix}
{\overset{r_1 - 4r_3}{\longrightarrow}}
\begin{pmatrix} 0 & 0 & 0\\ 0 & 1 & 0\\ 1 & 0 & 0\\ 0 & 0 & 1 \end{pmatrix}
\end{align*}
Here, the number of non-zero rows are three and hence the points are collinear which implies that the line passing through the given points are parallel.
\end{flushleft}

\end{document}
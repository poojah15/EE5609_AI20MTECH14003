\documentclass[journal,12pt,twocolumn]{IEEEtran}

\usepackage{setspace}
\usepackage{gensymb}

\singlespacing


\usepackage[cmex10]{amsmath}
%\usepackage{amsthm}
%\interdisplaylinepenalty=2500
%\savesymbol{iint}
%\usepackage{txfonts}
%\restoresymbol{TXF}{iint}
%\usepackage{wasysym}
\usepackage{amsthm}
%\usepackage{iithtlc}
\usepackage{mathrsfs}
\usepackage{txfonts}
\usepackage{stfloats}
\usepackage{bm}
\usepackage{cite}
\usepackage{cases}
\usepackage{subfig}
%\usepackage{xtab}
\usepackage{longtable}
\usepackage{multirow}
%\usepackage{algorithm}
%\usepackage{algpseudocode}
\usepackage{enumitem}
\usepackage{mathtools}
\usepackage{steinmetz}
\usepackage{tikz}
%\usepackage{circuitikz}
\usepackage{verbatim}
\usepackage{tfrupee}
\usepackage[breaklinks=true]{hyperref}
%\usepackage{stmaryrd}
\usepackage{tkz-euclide} % loads  TikZ and tkz-base
%\usetkzobj{all}
\usetikzlibrary{calc,math}
\usepackage{listings}
   \usepackage{color}                                            %%
    \usepackage{array}                                            %%
    \usepackage{longtable}                                        %%
    \usepackage{calc}                                             %%
    \usepackage{multirow}                                         %%
    \usepackage{hhline}                                           %%
    \usepackage{ifthen}                                           %%
  %optionally (for landscape tables embedded in another document): %%
    \usepackage{lscape}     
%\usepackage{multicol}
\usepackage{chngcntr}
%\usepackage{enumerate}

%\usepackage{wasysym}
%\newcounter{MYtempeqncnt}
\DeclareMathOperator*{\Res}{Res}
%\renewcommand{\baselinestretch}{2}
\renewcommand\thesection{\arabic{section}}
\renewcommand\thesubsection{\thesection.\arabic{subsection}}
\renewcommand\thesubsubsection{\thesubsection.\arabic{subsubsection}}

\renewcommand\thesectiondis{\arabic{section}}
\renewcommand\thesubsectiondis{\thesectiondis.\arabic{subsection}}
\renewcommand\thesubsubsectiondis{\thesubsectiondis.\arabic{subsubsection}}

% correct bad hyphenation here
\hyphenation{op-tical net-works semi-conduc-tor}
\def\inputGnumericTable{}                                 %%

\lstset{
%language=C,
frame=single, 
breaklines=true,
columns=fullflexible
}

\begin{document}
%


\newtheorem{theorem}{Theorem}[section]
\newtheorem{problem}{Problem}
\newtheorem{proposition}{Proposition}[section]
\newtheorem{lemma}{Lemma}[section]
\newtheorem{corollary}[theorem]{Corollary}
\newtheorem{example}{Example}[section]
\newtheorem{definition}[problem]{Definition}

\newcommand{\BEQA}{\begin{eqnarray}}
\newcommand{\EEQA}{\end{eqnarray}}
\newcommand{\define}{\stackrel{\triangle}{=}}
\bibliographystyle{IEEEtran}
%\bibliographystyle{ieeetr}
\providecommand{\mbf}{\mathbf}
\providecommand{\pr}[1]{\ensuremath{\Pr\left(#1\right)}}
\providecommand{\qfunc}[1]{\ensuremath{Q\left(#1\right)}}
\providecommand{\sbrak}[1]{\ensuremath{{}\left[#1\right]}}
\providecommand{\lsbrak}[1]{\ensuremath{{}\left[#1\right.}}
\providecommand{\rsbrak}[1]{\ensuremath{{}\left.#1\right]}}
\providecommand{\brak}[1]{\ensuremath{\left(#1\right)}}
\providecommand{\lbrak}[1]{\ensuremath{\left(#1\right.}}
\providecommand{\rbrak}[1]{\ensuremath{\left.#1\right)}}
\providecommand{\cbrak}[1]{\ensuremath{\left\{#1\right\}}}
\providecommand{\lcbrak}[1]{\ensuremath{\left\{#1\right.}}
\providecommand{\rcbrak}[1]{\ensuremath{\left.#1\right\}}}
\theoremstyle{remark}
\newtheorem{rem}{Remark}
\newcommand{\sgn}{\mathop{\mathrm{sgn}}}
\providecommand{\abs}[1]{\left\vert#1\right\vert}
\providecommand{\res}[1]{\Res\displaylimits_{#1}} 
\providecommand{\norm}[1]{\left\lVert#1\right\rVert}
%\providecommand{\norm}[1]{\lVert#1\rVert}
\providecommand{\mtx}[1]{\mathbf{#1}}
\providecommand{\mean}[1]{E\left[ #1 \right]}
\providecommand{\fourier}{\overset{\mathcal{F}}{ \rightleftharpoons}}
%\providecommand{\hilbert}{\overset{\mathcal{H}}{ \rightleftharpoons}}
\providecommand{\system}{\overset{\mathcal{H}}{ \longleftrightarrow}}
	%\newcommand{\solution}[2]{\textbf{Solution:}{#1}}
\newcommand{\solution}{\noindent \textbf{Solution: }}
\newcommand{\cosec}{\,\text{cosec}\,}
\providecommand{\dec}[2]{\ensuremath{\overset{#1}{\underset{#2}{\gtrless}}}}
\newcommand{\myvec}[1]{\ensuremath{\begin{pmatrix}#1\end{pmatrix}}}
\newcommand{\mydet}[1]{\ensuremath{\begin{vmatrix}#1\end{vmatrix}}}
%\numberwithin{equation}{section}
\numberwithin{equation}{subsection}
%\numberwithin{problem}{section}
%\numberwithin{definition}{section}
\makeatletter
\@addtoreset{figure}{problem}
\makeatother
\let\StandardTheFigure\thefigure
\let\vec\mathbf
%\renewcommand{\thefigure}{\theproblem.\arabic{figure}}
\renewcommand{\thefigure}{\theproblem}
%\setlist[enumerate,1]{before=\renewcommand\theequation{\theenumi.\arabic{equation}}
%\counterwithin{equation}{enumi}
%\renewcommand{\theequation}{\arabic{subsection}.\arabic{equation}}
\def\putbox#1#2#3{\makebox[0in][l]{\makebox[#1][l]{}\raisebox{\baselineskip}[0in][0in]{\raisebox{#2}[0in][0in]{#3}}}}
     \def\rightbox#1{\makebox[0in][r]{#1}}
     \def\centbox#1{\makebox[0in]{#1}}
     \def\topbox#1{\raisebox{-\baselineskip}[0in][0in]{#1}}
     \def\midbox#1{\raisebox{-0.5\baselineskip}[0in][0in]{#1}}
\vspace{3cm}
\title{Assignment-1}
\author{Pooja H \\ AI20MTECH14003}
\maketitle
\newpage
\bigskip
\renewcommand{\thefigure}{\theenumi}
\renewcommand{\thetable}{\theenumi}
\begin{abstract}
This assignment finds whether the lines passing through the given points are parallel or not.
\end{abstract}
Download all python codes from 

\begin{lstlisting}
svn co https://github.com/poojah15/EE5609_AI20MTECH14003
\end{lstlisting}


\section{Problem Statement}
To show that the line passing through the points $\myvec{4\\7\\8}, \myvec{2\\3\\4}$ is parallel to the line through the points $\myvec{-1\\-2\\1}, \myvec{1\\2\\5}$
\section{Theory}
Let the lines be parallel and the first two points pass through $\vec{n}^T\vec{x} = c1$\\
i.e.
\begin{align}
	\vec{n}^T\vec{x}_1=c_1 => \vec{x}_1^T\vec{n} = c_1 \\
	\vec{n}^T\vec{x}_2=c_2 => \vec{x}_2^T\vec{n} = c_2
\end{align}
and the second two points pass through $\vec{n}^T\vec{x} = c2$
Then
\begin{align}
	\vec{n}^T\vec{x}_3=c_3 => \vec{x}_3^T\vec{n} = c_3 \\
	\vec{n}^T\vec{x}_4=c_4 => \vec{x}_4^T\vec{n} = c_4
\end{align}
Putting all the equations together, we obtain
\begin{align}
	\myvec{\vec{x}_1^T\\ \vec{x}_2^T\\ \vec{x}_3^T\\ \vec{x}_4^T}\vec{n} = \myvec{c_1\\c_2\\c_3\\c_4}
\end{align}
Now if this equation has a solution, then $\vec{n}$ exists and the lines will be parallel.\\
\vspace{3mm}
\section{Example}
Given the points, $\vec{A} = \myvec{ 4\\7\\	8}, 
\vec{B} = \myvec{ 2\\ 3\\ 4}, and \quad
\vec{C} = \myvec{ -1\\ -2\\ 1 }, 
\vec{D} = \myvec{ 1\\ 2\\ 5} $\\

Applying the row reduction procedure on the coefficient matrix:
\begin{align*}
	\myvec{4 & 7 & 8\\ 2 & 3 & 4\\ -1 & -2 & 1\\ 1 & 2 & 5}
	\xleftrightarrow{r_3+r_4}\xleftrightarrow{r_1 - 2r_2}
	\myvec{4 & 7 & 8\\ 0 & 1 & 0\\ -1 & -2 & 1\\ 0 & 0 & 6}\\
	\xleftrightarrow{r_3-6r_4}\xleftrightarrow{r_1 - 7r_2}
	\myvec{4 & 0 & 8\\ 0 & 1 & 0\\ -1 & -2 & 0\\ 0 & 0 & 6}
	\xleftrightarrow{r_1-8r_4}\xleftrightarrow{r_4/6}
	\myvec{4 & 0 & 0\\ 0 & 1 & 0\\ -1 & -2 & 1\\ 0 & 0 & 1 } \\
    \xleftrightarrow{r_3+r_4}\xleftrightarrow{-r_3 - 2r_2}
	\myvec{4 & 0 & 0\\ 0 & 1 & 0\\ 1 & 0 & 0\\ 0 & 0 & 1}
	\xleftrightarrow{r_1 - 4r_3}
	\myvec{0 & 0 & 0\\ 0 & 1 & 0\\ 1 & 0 & 0\\ 0 & 0 & 1}
\end{align*}
Here, the number of non-zero rows are three and hence the rank of the matrix is 3 which implies that the solution exists. Therefore the lines passing through $\vec{A}, \vec{B}$ and $\vec{C}, \vec{D}$ are parallel.

\end{document}

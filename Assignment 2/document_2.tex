\documentclass[journal,12pt,twocolumn]{IEEEtran}

\usepackage{setspace}
\usepackage{gensymb}

\singlespacing


\usepackage[cmex10]{amsmath}
%\usepackage{amsthm}
%\interdisplaylinepenalty=2500
%\savesymbol{iint}
%\usepackage{txfonts}
%\restoresymbol{TXF}{iint}
%\usepackage{wasysym}
\usepackage{amsthm}
%\usepackage{iithtlc}
\usepackage{mathrsfs}
\usepackage{txfonts}
\usepackage{stfloats}
\usepackage{bm}
\usepackage{cite}
\usepackage{cases}
\usepackage{subfig}
%\usepackage{xtab}
\usepackage{longtable}
\usepackage{multirow}
%\usepackage{algorithm}
%\usepackage{algpseudocode}
\usepackage{enumitem}
\usepackage{mathtools}
\usepackage{steinmetz}
\usepackage{tikz}
%\usepackage{circuitikz}
\usepackage{verbatim}
\usepackage{tfrupee}
\usepackage[breaklinks=true]{hyperref}
%\usepackage{stmaryrd}
\usepackage{tkz-euclide} % loads  TikZ and tkz-base
%\usetkzobj{all}
\usetikzlibrary{calc,math}
\usepackage{listings}
   \usepackage{color}                                            %%
    \usepackage{array}                                            %%
    \usepackage{longtable}                                        %%
    \usepackage{calc}                                             %%
    \usepackage{multirow}                                         %%
    \usepackage{hhline}                                           %%
    \usepackage{ifthen}                                           %%
  %optionally (for landscape tables embedded in another document): %%
    \usepackage{lscape}     
%\usepackage{multicol}
\usepackage{chngcntr}
%\usepackage{enumerate}

%\usepackage{wasysym}
%\newcounter{MYtempeqncnt}
\DeclareMathOperator*{\Res}{Res}
%\renewcommand{\baselinestretch}{2}
\renewcommand\thesection{\arabic{section}}
\renewcommand\thesubsection{\thesection.\arabic{subsection}}
\renewcommand\thesubsubsection{\thesubsection.\arabic{subsubsection}}

\renewcommand\thesectiondis{\arabic{section}}
\renewcommand\thesubsectiondis{\thesectiondis.\arabic{subsection}}
\renewcommand\thesubsubsectiondis{\thesubsectiondis.\arabic{subsubsection}}

% correct bad hyphenation here
\hyphenation{op-tical net-works semi-conduc-tor}
\def\inputGnumericTable{}                                 %%

\lstset{
%language=C,
frame=single, 
breaklines=true,
columns=fullflexible
}

\begin{document}
%


\newtheorem{theorem}{Theorem}[section]
\newtheorem{problem}{Problem}
\newtheorem{proposition}{Proposition}[section]
\newtheorem{lemma}{Lemma}[section]
\newtheorem{corollary}[theorem]{Corollary}
\newtheorem{example}{Example}[section]
\newtheorem{definition}[problem]{Definition}

\newcommand{\BEQA}{\begin{eqnarray}}
\newcommand{\EEQA}{\end{eqnarray}}
\newcommand{\define}{\stackrel{\triangle}{=}}
\bibliographystyle{IEEEtran}
%\bibliographystyle{ieeetr}
\providecommand{\mbf}{\mathbf}
\providecommand{\pr}[1]{\ensuremath{\Pr\left(#1\right)}}
\providecommand{\qfunc}[1]{\ensuremath{Q\left(#1\right)}}
\providecommand{\sbrak}[1]{\ensuremath{{}\left[#1\right]}}
\providecommand{\lsbrak}[1]{\ensuremath{{}\left[#1\right.}}
\providecommand{\rsbrak}[1]{\ensuremath{{}\left.#1\right]}}
\providecommand{\brak}[1]{\ensuremath{\left(#1\right)}}
\providecommand{\lbrak}[1]{\ensuremath{\left(#1\right.}}
\providecommand{\rbrak}[1]{\ensuremath{\left.#1\right)}}
\providecommand{\cbrak}[1]{\ensuremath{\left\{#1\right\}}}
\providecommand{\lcbrak}[1]{\ensuremath{\left\{#1\right.}}
\providecommand{\rcbrak}[1]{\ensuremath{\left.#1\right\}}}
\theoremstyle{remark}
\newtheorem{rem}{Remark}
\newcommand{\sgn}{\mathop{\mathrm{sgn}}}
\providecommand{\abs}[1]{\left\vert#1\right\vert}
\providecommand{\res}[1]{\Res\displaylimits_{#1}} 
\providecommand{\norm}[1]{\left\lVert#1\right\rVert}
%\providecommand{\norm}[1]{\lVert#1\rVert}
\providecommand{\mtx}[1]{\mathbf{#1}}
\providecommand{\mean}[1]{E\left[ #1 \right]}
\providecommand{\fourier}{\overset{\mathcal{F}}{ \rightleftharpoons}}
%\providecommand{\hilbert}{\overset{\mathcal{H}}{ \rightleftharpoons}}
\providecommand{\system}{\overset{\mathcal{H}}{ \longleftrightarrow}}
	%\newcommand{\solution}[2]{\textbf{Solution:}{#1}}
\newcommand{\solution}{\noindent \textbf{Solution: }}
\newcommand{\cosec}{\,\text{cosec}\,}
\providecommand{\dec}[2]{\ensuremath{\overset{#1}{\underset{#2}{\gtrless}}}}
\newcommand{\myvec}[1]{\ensuremath{\begin{pmatrix}#1\end{pmatrix}}}
\newcommand{\mydet}[1]{\ensuremath{\begin{vmatrix}#1\end{vmatrix}}}
%\numberwithin{equation}{section}
\numberwithin{equation}{subsection}
%\numberwithin{problem}{section}
%\numberwithin{definition}{section}
\makeatletter
\@addtoreset{figure}{problem}
\makeatother
\let\StandardTheFigure\thefigure
\let\vec\mathbf
%\renewcommand{\thefigure}{\theproblem.\arabic{figure}}
\renewcommand{\thefigure}{\theproblem}
%\setlist[enumerate,1]{before=\renewcommand\theequation{\theenumi.\arabic{equation}}
%\counterwithin{equation}{enumi}
%\renewcommand{\theequation}{\arabic{subsection}.\arabic{equation}}
\def\putbox#1#2#3{\makebox[0in][l]{\makebox[#1][l]{}\raisebox{\baselineskip}[0in][0in]{\raisebox{#2}[0in][0in]{#3}}}}
     \def\rightbox#1{\makebox[0in][r]{#1}}
     \def\centbox#1{\makebox[0in]{#1}}
     \def\topbox#1{\raisebox{-\baselineskip}[0in][0in]{#1}}
     \def\midbox#1{\raisebox{-0.5\baselineskip}[0in][0in]{#1}}
\vspace{3cm}
\title{Assignment-2}
\author{Pooja H \\ AI20MTECH14003}
\maketitle
\newpage
\bigskip
\renewcommand{\thefigure}{\theenumi}
\renewcommand{\thetable}{\theenumi}
\begin{abstract}
In this work, we compute the modulus (norm) of the complex numbers.
\end{abstract}
Download all python codes from 
\begin{lstlisting}
	https://github.com/poojah15/EE5609_AI20MTECH14003
\end{lstlisting}
Download all latex-tikz codes from 
\begin{lstlisting}
	https://github.com/poojah15/EE5609_AI20MTECH14003
\end{lstlisting}


\section{Problem Statement}
If $\vec{z}_1 = \myvec{2 \\ -1}, \vec{z}_2 = \myvec{1 \\ 1}$, find $\norm{\frac{\vec{z}_1 + \vec{z}_1 + 1}{\vec{z}_1-\vec{z}_2+1}}$ 

\section{Solution}
Let us consider $\frac{\vec{z}_1 + \vec{z}_1 + 1}{\vec{z}_1-\vec{z}_2+1}$, then
\begin{align}
	\vec{z}_1 + \vec{z}_1 + 1 &= \myvec{2 \\ -1} + \myvec{ 2 \\ -1} + \myvec{ 1 \\ 0} \\
                  &= \myvec{5 \\ -2}\\
	\vec{z}_1 - \vec{z}_2 + 1 &= \myvec{2 \\ -1} - \myvec{ 1 \\ 1} + \myvec{ 1 \\ 0} \\
	&= \myvec{2 \\ -2} \\ 
 	\frac{\vec{z}_1 + \vec{z}_1 + 1}{\vec{z}_1-\vec{z}_2+1} &= \frac{\myvec{5 \\ -2}}{\myvec{2 \\ -2}}\label{eq:result} 
\end{align}
In general, the complex number $\myvec{a_1 \\ a_2}$ can be represented in the form of matrix as:
\begin{align} \label{eq:matrix_form}
	\myvec{a_1 \\ a_2} &= \myvec{ a_1 & -a_2 \\ a_2 & a_1}\myvec{1 \\ 0}
\end{align}
Therefore using \eqref{eq:matrix_form}, \eqref{eq:result} can be represented as:
\begin{align}
\frac{\vec{z}_1 + \vec{z}_1 + 1}{\vec{z}_1-\vec{z}_2+1}	&=\myvec{5 & 2\\ -2 & 5} \myvec{2 & 2\\ -2 & 2} ^{-1} \myvec{1 \\ 0}\\
	&= \myvec{5 & 2\\ -2 & 5} \myvec{1/4 & -1/4 \\ 1/4 & 1/4} \myvec{1 \\ 0}\\
	&= \myvec{7/4 & -3/4 \\ 3/4 & 7/4} \myvec{1\\0} \label{eq:res2} \end{align}
Using \eqref{eq:matrix_form} we get,
\begin{align}
\frac{\vec{z}_1 + \vec{z}_1 + 1}{\vec{z}_1-\vec{z}_2+1}	&= \myvec{7/4 \\ 3/4}
\end{align}
The modulus of a complex number $\myvec{a \\ b}$ is defined as $\sqrt{a^{2} + b^{2}}$.
Therefore, 
\begin{align}
	\norm{\frac{\vec{z}_1 + \vec{z}_1 + 1}{\vec{z}_1-\vec{z}_2+1}} &= \sqrt{(7/4)^{2} + (3/4)^{2}} \\ 
	&= \frac{\sqrt{58}}{4}
\end{align}
\end{document}

\documentclass[a4paper,12pt]{article}
\usepackage{amsmath, mathtools}
\usepackage{extarrows}
\newcommand{\myvec}[1]{\ensuremath{\begin{pmatrix}#1\end{pmatrix}}}

\begin{document}
\begin{flushleft}	
\textbf{Problem Statement:} \\
\textit{To verify whether the lines passing through the given set of points are parallel or not} \vspace{5mm}

\textbf{Solution-1}
\textit{Using the vector representation} \\
Given the points, $A = \begin{pmatrix} 4\\	7\\	8\end{pmatrix}, 
                   B = \begin{pmatrix} 2\\ 3\\ 4 \end{pmatrix}, and \quad
                   C = \begin{pmatrix} -1\\ -2\\ 1 \end{pmatrix}, 
                   D = \begin{pmatrix} 1\\ 2\\ 5 \end{pmatrix} $


\textbullet{Compute the direction vector for the given set of points}
\begin{align}
B-A = \begin{pmatrix} -2\\ -4\\ -4 \end{pmatrix}   \\
D-C = \begin{pmatrix}  2\\  4\\  4 \end{pmatrix}  
\end{align}

\textbullet{Check whether one of the direction vector is the scalar multiple of the other direction vector}\\
Here, from (1) and (2), $B-A = k (D-C)$. In this example, k = -1.
\textbf{Hence, the lines are parallel.}\\
\vspace{5mm}
\textbf{Solution-2} \textit{Using the matrix representation and rank of a matrix}\\
Represent the direction vectors in the matrix form and perform row reduction: 
\begin{align*}
i.e., M = (B-A \quad D-C)^T \\
M = \begin{pmatrix}
	-2 & -4 & -4\\
	 2 &  4 &  4
    \end{pmatrix}
\xleftrightarrow{R_2 \leftarrow R_1+R_2}
\begin{pmatrix}
	-2 & -4 & -4\\
	 0 &  0 &  0
\end{pmatrix}
\end{align*}

Here, the rank of the matrix is 1. This implies that the lines are parallel.\\
\vspace{5mm}
\textbf{Solution-3} \textit{Using the cross product of the vectors}\\
\textbullet{Compute the cross product of the direction vectors}\\
The cross product of the direction vectors given in (1) and (2) is: 
\begin{align*}
\begin{pmatrix} -2\\ -4\\ -4 \end{pmatrix} X
\begin{pmatrix}  2\\  4\\  4 \end{pmatrix} =
\begin{pmatrix}
 -16 + 16 \\
 -8+8 \\
 -8+8\\
\end{pmatrix}
=
\begin{pmatrix}
0 \\
0 \\
0\\
\end{pmatrix}
\end{align*}

The zero vector infers that the lines are parallel.\\
\vspace{5mm}
\textbf{Solution-4} 
Let the lines be parallel and the first two points pass through $n^T\textbf{x} = \textbf{c}$
i.e.
\begin{align}
n^Tx1=c1 => x1^Tn = c1, \quad
n^Tx2=c2 => x2^Tn = c2
\end{align}
and the second two points pass through $n^T\textbf{x} = \textbf{c}$
Then
\begin{align}
n^Tx3=c3 => x3^Tn = c3, \quad
n^Tx4=c4 => x4^Tn = c4
\end{align}
Putting equations (3) and (4) together, we obtain
\begin{align}
\myvec{x1^T\\x2^T\\x3^T\\x4^T}\vec{n} = \myvec{c1\\c2\\c3\\c4}
\end{align}
Now if this equation has a solution, then $\vec{n}$ exists and the lines will be parallel.
\end{flushleft}
\end{document}